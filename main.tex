\documentclass[letterpaper]{article}
\usepackage{graphicx}
\usepackage{mathtools}
\usepackage{tabularx}
\usepackage{hyperref}
\usepackage{float}
\usepackage{parskip}
\usepackage{color,soul}
\usepackage{amsfonts}
\usepackage{pdfpages}
\usepackage{xfrac}
\usepackage{fix-cm}
\usepackage{bookmark}
\usepackage{listings}
\usepackage{framed}
\usepackage{makecell}
\usepackage[dvipsnames]{xcolor}
\setlength{\parskip}{1em}
\parindent=0pt

\title{ESE5400: Case Study \#1}
\author{Calum Mitchell, Siddharth Ramanathan, Vishnu Venkatesh}
\date{October 2024}

\begin{document}

\maketitle

The risk-free rate should be the same one as the expected investment horizon - a typical value of 10 years should be the one we consider, so the risk free rate that we are using in the CAPM model is 4.66\%. With $r_f = 4.66\%$, we can compute the cost of debt $r_D$ and the cost of equity $r_E$.

For the consolidated company, Midland's pre-tax cost of debt can be computed from the risk-free rate $r_f$ and the spread to treasury.

\begin{table}[H]
    \begin{tabularx}{\textwidth}{|>{\centering\arraybackslash}X|>{\centering\arraybackslash}X|>{\centering\arraybackslash}X|>{\centering\arraybackslash}X|}
        \hline
        Business Segment & Credit Rating & Spread to Treasury & Cost of debt \\ \hline
        Consolidated & A+ & 1.62\% & 6.28\%\\ \hline
        Exploration \& Production & A+ & 1.60\% & 6.26\%\\ \hline
        Refining \& Marketing & BBB & 1.80\% & 6.46\% \\ \hline
        Petrochemicals & AA- & 1.35\% & 6.01\% \\ \hline
    \end{tabularx}
\end{table}

The cost of debt (before tax) would be for the consolidated company, so

\[
r_D = 6.28\%
\]

From Exhibit 1, it would make the most sense to use the most recent tax rate for WACC calculation. This reflects the company's present tax situation and is the most relevant for future financial analyses.

\begin{align*}
\text{Effective Tax Rate} &= \frac{\text{Taxes}}{\text{Income before taxes}} \times 100 \\
&= \frac{\$11,747}{\$30,446} \times 100 \\
t &= \text{38.58\%}
\end{align*}

The EMRP represents the difference between the expected return on equity and the risk-free rate. 

\[
\text{EMRP} = \text{Expected Equity Return} - \text{Risk-Free Rate}
\]

An EMRP of 5\% implies the expected equity return be 4.66\% + 5\% = 9.66\%. We can use this to estimate Midland's consolidated cost of equity using the formula below.

\[
r_E = r_f + \beta \times \text{EMRP}
\]

We can get the value of $\beta$ from Exhibit 5 - the equity beta is given to be 1.25.

\begin{align*}
r_E &= 4.66\% + 1.25 \times 5\% \\
r_E &= 10.91\%
\end{align*}

To calculate the after-tax cost of capital, we need one more piece of information - the debt/value $\lambda$, which we can see from Table 1 is 42.2\% for the consolidated company.

\begin{align*}
WACC_{a.t.} &= \lambda (1-t) r_D + (1-\lambda)r_E \\
            &= 0.0422 \times (1 - 0.3858) \times 0.0628 + (1 - 0.0422) \times 0.1091 \\
            &= 0.1061
\end{align*}

The $WACC_{a.t.}$ is 10.61\%.

\end{document}

\documentclass[letterpaper]{article}
\usepackage{graphicx}
\usepackage{mathtools}
\usepackage{tabularx}
\usepackage{hyperref}
\usepackage{float}
\usepackage{parskip}
\usepackage{color,soul}
\usepackage{amsfonts}
\usepackage{pdfpages}
\usepackage{xfrac}
\usepackage{fix-cm}
\usepackage{bookmark}
\usepackage{listings}
\usepackage{framed}
\usepackage[dvipsnames]{xcolor}
\setlength{\parskip}{1em}
\parindent=0pt

\title{ESE5400: Case Study \#1}
\author{Calum Mitchell, Siddharth Ramanathan, Vishnu Venkatesh}
\date{October 2024}

\begin{document}

\maketitle

The risk-free rate should be the same one as the expected investment horizon - a typical value of 10 years should be the one we consider, so the risk free rate that we are using in the CAPM model is 4.66\%. With $r_f = 4.66\%$, we can compute the cost of debt $r_D$ and the cost of equity $r_E$.

\end{document}

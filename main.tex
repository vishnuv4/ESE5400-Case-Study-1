\documentclass[letterpaper]{article}
\usepackage{graphicx}
\usepackage{mathtools}
\usepackage{tabularx}
\usepackage{hyperref}
\usepackage{float}
\usepackage{parskip}
\usepackage{color,soul}
\usepackage{amsfonts}
\usepackage{pdfpages}
\usepackage{xfrac}
\usepackage{fix-cm}
\usepackage{bookmark}
\usepackage{listings}
\usepackage{framed}
\usepackage{makecell}
\usepackage[dvipsnames]{xcolor}
\setlength{\parskip}{1em}
\parindent=0pt

\title{ESE5400: Case Study \#1}
\author{Calum Mitchell, Siddharth Ramanathan, Vishnu Venkatesh}
\date{October 2024}

\begin{document}

\maketitle

The risk-free rate should be the same one as the expected investment horizon - a typical value of 10 years should be the one we consider, so the risk free rate that we are using in the CAPM model is 4.66\%. With $r_f = 4.66\%$, we can compute the cost of debt $r_D$ and the cost of equity $r_E$.

For the consolidated company, Midland's pre-tax cost of debt can be computed from the risk-free rate $r_f$ and the spread to treasury.

\begin{table}[H]
    \begin{tabularx}{\textwidth}{|>{\centering\arraybackslash}X|>{\centering\arraybackslash}X|>{\centering\arraybackslash}X|>{\centering\arraybackslash}X|}
        \hline
        Business Segment & Credit Rating & Spread to Treasury & Cost of debt \\ \hline
        Consolidated & A+ & 1.62\% & 6.28\%\\ \hline
        Exploration \& Production & A+ & 1.60\% & 6.26\%\\ \hline
        Refining \& Marketing & BBB & 1.80\% & 6.46\% \\ \hline
        Petrochemicals & AA- & 1.35\% & 6.01\% \\ \hline
    \end{tabularx}
\end{table}

\end{document}

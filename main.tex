\documentclass[letterpaper]{article}
\usepackage{graphicx}
\usepackage{mathtools}
\usepackage{tabularx}
\usepackage{hyperref}
\usepackage{float}
\usepackage{parskip}
\usepackage{color,soul}
\usepackage{amsfonts}
\usepackage{pdfpages}
\usepackage{xfrac}
\usepackage{fix-cm}
\usepackage{bookmark}
\usepackage{listings}
\usepackage{framed}
\usepackage{makecell}
\usepackage[dvipsnames]{xcolor}
\setlength{\parskip}{1em}
\parindent=0pt

\title{ESE5400: Case Study \#1}
\author{Calum Mitchell, Siddharth Ramanathan, Vishnu Venkatesh}
\date{October 2024}

\begin{document}

\maketitle

The risk-free rate should be the same one as the expected investment horizon - a typical value of 10 years should be the one we consider, so the risk free rate that we are using in the CAPM model is 4.66\%. With $r_f = 4.66\%$, we can compute the cost of debt $r_D$ and the cost of equity $r_E$.

For the consolidated company, Midland's pre-tax cost of debt can be computed from the risk-free rate $r_f$ and the spread to treasury.

\begin{table}[H]
    \begin{tabularx}{\textwidth}{|>{\centering\arraybackslash}X|>{\centering\arraybackslash}X|>{\centering\arraybackslash}X|>{\centering\arraybackslash}X|}
        \hline
        Business Segment & Credit Rating & Spread to Treasury & Cost of debt \\ \hline
        Consolidated & A+ & 1.62\% & 6.28\%\\ \hline
        Exploration \& Production & A+ & 1.60\% & 6.26\%\\ \hline
        Refining \& Marketing & BBB & 1.80\% & 6.46\% \\ \hline
        Petrochemicals & AA- & 1.35\% & 6.01\% \\ \hline
    \end{tabularx}
\end{table}

The cost of debt (before tax) would be for the consolidated company, so

\[
r_D = 6.28\%
\]

These 4 costs of debt are different because the credit ratings correspond with the perceived amount of risk involved in the debt. For example, a rating of BBB is the lowest credit rating on the table and we see that it has the highest spread (and therefore the highest cost of debt) of 1.80\% compared to the other business segments in the table. Conversely, AA- is the highest credit rating and therefore the least risky to loan money to - and we see that it has a correspondingly lower spread than any other business segment at 1.35\%. 

We can also analyze the difference in the 4 costs of debt based on the nature of operations. Refining \& Marketing is seen as a riskier venture to invest in for three reasons: (a) stiff competition; (b) highly commoditized products; and (c) declining margins over the previous 20 years. So Refining \& Marketing has the lowest credit rating and therefore the highest cost of debt. Exploration \& Production was one of Midland's most profitable ventures, so there is a little more confidence in that business venture. However, that is offset by the fact that Midland has assets in politically volatile regions like the Middle East, Central Asia, Russia, and West Africa. So it has a slightly higher credit rating of A+. Petrochemicals has the highest credit rating because it has manufacturing facilities and research centers all over the world. Several older facilities were also replaced by more efficient ones - so there is much more confidence in this business segment compared to the other two, and that reflects in the highest credit rating out of the three (AA-) and the lowest cost of debt (6.01\%).

For Midland's WACC calculations, it makes the most sense to take the average of the three tax rates in 2004, 2005, and 2006.

\begin{align*}
    t_{2004} = \frac{\$7,414}{\$17,910} &= 41.40\% \\
    t_{2004} = \frac{\$7,414}{\$17,910} &= 39.20\% \\
    t_{2004} = \frac{\$7,414}{\$17,910} &= 38.58\% \\
    t_{avg} = \frac{41.40\% + 39.20\% + 38.58\%}{3} &= 39.73\%
\end{align*}

So the average tax rate that we recommend using for WACC calculations is 39.73\%.

The EMRP represents the difference between the expected return on equity and the risk-free rate. 

\[
\text{EMRP} = \text{Expected Equity Return} - \text{Risk-Free Rate}
\]

An EMRP of 5\% implies the expected equity return be 4.66\% + 5\% = 9.66\%. We can use this to estimate Midland's consolidated cost of equity using the formula below.

\[
r_E = r_f + \beta \times \text{EMRP}
\]

We can get the value of $\beta$ from Exhibit 5 - the equity beta is given to be 1.25.

\begin{align*}
r_E &= 4.66\% + 1.25 \times 5\% \\
r_E &= 10.91\%
\end{align*}

To calculate the after-tax cost of capital, we need one more piece of information - the debt/value $\lambda$. We see from Exhibit 5 that the D/E ratio is 59.3\%.

\begin{align*}
    \lambda &= \frac{D}{D+E} \\
            &= \frac{\frac{D}{E}}{\frac{D}{E} + 1} \\
            &= \frac{0.593}{0.593 + 1} \\
            &= \frac{0.593}{1.593} \\
    \lambda &= 37.23\%
\end{align*}

\begin{align*}
WACC_{a.t.} &= \lambda (1-t) r_D + (1-\lambda)r_E \\
            &= 0.3723 \times (1 - 0.3973) \times 0.0628 + (1 - 0.3723) \times 0.1091 \\
            &= 0.08258
\end{align*}

The $WACC_{a.t.}$ is 8.26\%.

To find the un-levered asset beta, we can use the formula below.

\begin{align*}
\beta_{\text{Asset}} &= \frac{\text{Equity}}{\text{Debt + Equity}}\times \beta_{\text{Equity}} \\
                    &= (1 - \lambda) \times \beta_{\text{Equity}} \\
                    &= (1 - 0.3723) \times 1.25 \\
                    &= 0.78468
\end{align*}

The un-levered asset beta is 0.785.

\end{document}
